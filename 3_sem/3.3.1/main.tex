\documentclass{article}

% Russian language
\usepackage[utf8]{inputenc}
\usepackage[russian]{babel}

\usepackage{amsmath, amssymb}
\usepackage[left=4cm, right=4cm, top=2cm]{geometry}
\usepackage{array}

\usepackage{graphicx}
\usepackage{ragged2e}
\usepackage{wrapfig}
\justifying

\title{
    \textbf{Лабораторная работа 3.3.1}
}
\author{Герасименко Д.В.}
\date{2 курс ФРКТ, группа Б01-104}

\begin{document}
\maketitle

\begin{center}
    \raggedleft
        \underline{\underline{\LARGE {Аннотация}}}
\end{center}

\begin{center}
\raggedright
    \large{\textbf{Тема:}}
    \\
    \large {Измерение удельного заряда электрона методами магнитной фокусировки и магнетрона}
    
    \large{\textbf{Цель работы:}}
    \\
    \large {Определение значения магнитных полей, при которых происходит фокусировка электронного пучка, и по результатам измерений считать удельный заряд электрона $e/m$}
    
    \large{\textbf{Оборудование:}}
    \\
    \large{ A) Электронно-лучевая трубка и блок питания к ней; источник постоянного тока; соленоид; электростатический вольтметр; милливеберметр; ключи. B) Электронная лампа с цилиндрическим анодом; соленоид; источники питания лампы и соленоида; вольтметр постоянного тока; миллиамперметр, амперметр.}
\end{center}

\begin{center}
    \underline{\large {А. Метод магнитной фокусировки}}
\end{center}

\subsection* {Теория:}
Движение заряженной частицы в однородном магнитном представляет собой спирали, радиус которых определяется формулой: \(R = \frac{mv}{qB} = \frac{v}{\omega_{B}}\). За время циклотронного периода (\(T_{B} = \frac{2\pi R_{B}}{v_{\perp}}\)) частица преодолеет расстояние \(L\) (шаг спирали):
\begin{equation*}
    L = v_{\parallel} T_{B} = \frac{2\pi v cos(\alpha)}{\frac{e}{m} B}
\end{equation*}
где \(\alpha\) - угол в \([\overrightarrow{v},\overrightarrow{B}]\), соответственно при малых углах \(\alpha\) верно, что после оборота все электроны, вышедшие из одной точки, сфокусируются: \(L \approx \frac{2\pi v}{\frac{e}{m} B}\).
В данном методе электроны разгоняются анодным напряжением \(U\), фокусировка происходит при разных шагах \(\frac{L}{n}\), откуда получается исходная формула:
\begin{equation}
    \frac{e}{m} = \frac{8 \pi^{2} U}{L^{2}} \frac{n^{2}}{B^{2}}
\end{equation}

\subsection* {Экспериментальная установка:}

Основной частью установки является электронный осциллограф, трубка которого вынута и установлена в длинном соленоиде, создающим магнитное поле. Напряжение на отклоняющие пластины и питание подводятся к трубке многожильным кабелем.

\begin{wrapfigure}{r}{0.2\textwidth}
    \centering
    \includegraphics[width=0.25\textwidth]{images/image1.png}
    
    рис.1. Схема установки
\end{wrapfigure}

Пучок электронов, вылетающих из катода с разными скоростями, ускоряется анодным напряжением. Пропустив пучок сквозь две узкие диафрагмы, можно выделить электроны с практически одинаковой продольной скоростью. Небольшое переменное напряжение, поступающее с клеммы "Контрольный сигнал" осциллографа на отклоняющие пластины, изменяет только поперечную составляющую скорости. При увеличении магнитного поля линия на экране стягивается в точку, а затем снова удлиняется. 

Магнитное поле создается постоянным током, величина которого регулируется ручками источника питания и измеряется амперметром. Ключ служит для изменения направления поля в соленоиде.

Величина магнитного поля определяется с помощью милливеберметра.

На точность результатов может влиять внешнее магнитное поле, особенно продольное. 

Измерения магнитного поля с помощью милливеберметра обычно проводятся в предварительных опыта: при отключении ключа устанавливается связь между силой тока и индукцией магнитного поля в соленоиде. 

\subsection* {Ход работы:}
1) Определим связь между индукцией \(B\) магнитного поля и током \(I\) через обмотку магнита. Снимем зависимость магнитного потока: Ф \(= BSN\) в прямом и обратном направлениях. Погрешности прямых измерений: \(\sigma_{I} = 0,01 A; \sigma_{\text{Ф}} = 0,01 \text{мВб}\)

\begin{center}
\begin{tabular}{|c|c|}
    \hline
    $I$, A & $\Phi$, мВб \\ \hline
    0,26 & 0,35 \\ \hline
    0,41 & 0,50 \\ \hline
    0,79 & 1,05 \\ \hline
    1,01 & 1,35 \\ \hline
    1,49 & 1,95 \\ \hline
    2,28 & 3,00 \\ \hline
    2,54 & 3,30 \\ \hline
    3,06 & 3,90 \\ \hline
    3,64 & 4,50 \\ \hline
\end{tabular}\\
  \textbf{Таблица 1.} Зависимость $\Phi (I)$ в прямом направлении.\\
  \end{center}

\begin{center}
    \includegraphics[width = 0.7\textwidth]{images/image2.png}\\
    \textbf{График 1.} $\Phi (I)$ в прямом направлении.
\end{center}
  
\begin{center}
\begin{tabular}{|c|c|}
\hline
$I$, A & $\Phi$, мВб\\ \hline
0,26 & 4,66 \\ \hline
0,41 & 4,47 \\ \hline
0,79 & 3,98 \\ \hline
1,01 & 3,69 \\ \hline
1,49 & 3,07 \\ \hline
2,28 & 2,05 \\ \hline
2,54 & 1,71 \\ \hline
3,06 & 1,03 \\ \hline
3,64 & 0,28 \\ \hline
\end{tabular}\\
\textbf{Таблица 2.} Зависимость $\Phi (I)$ в обратном направлении.
\end{center} 
\begin{center}
    \includegraphics[width = 0.7\textwidth]{images/image3.png}\\
    \textbf{График 2.} $\Phi (I)$ в обратном направлении.
\end{center}

2) Постепенно увеличивая ток, зафиксируем значения, при которых наблюдается фокус. По ним определеим соответствующие значения величины магнитного поля, исходя из калибровки прибора. 

\begin{center}
    \begin{tabular}{|c|c|c|c|c|c|}
    \hline
    N   &   1   &   2   &   3   &   4   &   5  \\ \hline
    \(I_{\Phi}\), A & 0,53 & 1,12 & 1,7 & 2,34 & 2,89 \\    \hline
    \(B_{\Phi}\), мТл & 2,24 & 4,74 & 7,2 & 9,9 & 12,2 \\   \hline
    \end{tabular}
    \\
    \textbf{Таблица 3.} Зависимость \(B_{\Phi} = f(I)\) в прямом направлении
\end{center}

\begin{center}
    \includegraphics[width = 0.7\textwidth]{images/image4.png}\\
    \textbf{График 3.} $B_{\Phi} = f(I)$ в прямом направлении.
\end{center}

Аналогично для тока в обратном направлении:

\begin{center}
    \begin{tabular}{|c|c|c|c|c|c|}
    \hline
    N   &   1   &   2   &   3   &   4   &   5  \\ \hline
    \(I_{\Phi}\), A & 0,58 & 1,15 & 1,8 & 2,36 & 2,86 \\    \hline
    \(B_{\Phi}\), мТл & 14,2 & 11,7 & 8,9 & 6,5 & 4,3 \\   \hline
    \end{tabular}
    \\
    \textbf{Таблица 4.} Зависимость \(B_{\Phi} = f(I)\) в обратном направлении
\end{center}

\begin{center}
    \includegraphics[width = 0.7\textwidth]{images/image5.png}\\
    \textbf{График 4.} $B_{\Phi} = f(I)$ в обратном направлении.
\end{center}

В итоге, подставив в формулу $(1)$ мы получаем, что 
\[\dfrac{e}{m} = \left(1,6 \pm 0,2\right) \cdot 10^{11} \text{Кл}/\text{кг}\]
\\
\begin{center}
    \underline{\large {В. Метод магнитной фокусировки}}
\end{center}

\subsection*{Теория:}
Здесь удельный заряд электрона определяется по формуле
\begin{equation}
\dfrac{e}{m_e} = \dfrac{8V_a}{B_{\text{кр}}^2r_a^2},
\end{equation}
где $V_a$ - анодное напряжение, $B_{\text{кр}}$ - критическое поле, $r_a$ - радиус анода.
\subsection*{Описание установки.}
\begin{wrapfigure}{l}{0.3\textwidth}
  \begin{center}
    \includegraphics[width = 0.3\textwidth]{images/image6.png}
  \end{center}
  Рис 2. Схема установки.
\end{wrapfigure}
Два крайних цилиндра изолированы от среднего небольшими зазорами и используются для устранения краевых эффектов на торцах среднего цилиндра, ток с которого используется при измерениях. В качестве катода используется тонкая вольфрамовая проволока. Катод разогревается переменным током, отбираемым от стабилизированного источника питания. 

С этого же источника на анод лампы подается напряжение, регулируемое с помощью потенциометра и измеряемое вольтметром.

Индукция магнитного поля в соленоиде рассчитывается по току $I_m$, протекающему через обмотку соленоида. Коэффициент пропорциональности между ними указан в установке.

Лампа закреплена в соленоиде. Магнитное поле в соленоиде создается постоянным током, сила которого регулируется ручками источника питания и измеряется амперметром.

\subsection*{Ход работы.}
1) Параметры установки: \\
-   коэффициент пропорциональности между анодным током и магнитной индукцией \(B - K = 2,8 \cdot 10^{2}\) Тл/A; \\
-   радиус анода \(r_{A} = 12\) мм; \\
-   \(\sigma_{I_{m}} = 4 \text{мА}, \sigma_{B} = 0,1 \text{мТл}, \sigma_{I_{a}} = 2 \text{мкА}\); \\
Снимем зависимость анодного тока от тока через соленоид для различных значений \(V_{a}\)

\begin{center}
\begin{tabular}{|c|c|c|c|c|c|}
\hline
$I_m$, мА & $\sigma_{I_m}$, мА & $B$, мТл & $\sigma_B$, мТл & $I_a$, мкА & $\sigma_{I_a}$, мкА \\ \hline
0 & 4 & 0,0 & 0,1 & 266 & 2 \\ \hline
20 & 4 & 0,6 & 0,1 & 270 & 2 \\ \hline
32 & 4 & 0,9 & 0,1 & 266 & 2 \\ \hline
36 & 4 & 1,0 & 0,1 & 266 & 2 \\ \hline
44 & 4 & 1,2 & 0,1 & 266 & 2 \\ \hline
60 & 4 & 1,7 & 0,1 & 266 & 2 \\ \hline
76 & 4 & 2,1 & 0,1 & 262 & 2 \\ \hline
88 & 4 & 2,5 & 0,1 & 256 & 2 \\ \hline
96 & 4 & 2,7 & 0,1 & 252 & 2 \\ \hline
100 & 4 & 2,8 & 0,1 & 242 & 2 \\ \hline
108 & 4 & 3,0 & 0,1 & 236 & 2 \\ \hline
116 & 4 & 3,2 & 0,1 & 232 & 2 \\ \hline
124 & 4 & 3,5 & 0,1 & 230 & 2 \\ \hline
132 & 4 & 3,7 & 0,1 & 232 & 2 \\ \hline
136 & 4 & 3,8 & 0,1 & 222 & 2 \\ \hline
144 & 4 & 4,0 & 0,1 & 206 & 2 \\ \hline
148 & 4 & 4,1 & 0,1 & 200 & 2 \\ \hline
156 & 4 & 4,4 & 0,1 & 186 & 2 \\ \hline
160 & 4 & 4,5 & 0,1 & 180 & 2 \\ \hline
168 & 4 & 4,7 & 0,1 & 164 & 2 \\ \hline
176 & 4 & 4,9 & 0,1 & 140 & 2 \\ \hline
182 & 4 & 5,1 & 0,1 & 104 & 2 \\ \hline
184 & 4 & 5,2 & 0,1 & 88 & 2 \\ \hline
188 & 4 & 5,3 & 0,1 & 56 & 2 \\ \hline
190 & 4 & 5,3 & 0,1 & 40 & 2 \\ \hline
194 & 4 & 5,4 & 0,1 & 28 & 2 \\ \hline
202 & 4 & 5,7 & 0,1 & 18 & 2 \\ \hline
208 & 4 & 5,8 & 0,1 & 12 & 2 \\ \hline
216 & 4 & 6,0 & 0,1 & 10 & 2 \\ \hline
224 & 4 & 6,3 & 0,1 & 6 & 2 \\ \hline
236 & 4 & 6,6 & 0,1 & 4 & 2 \\ \hline
244 & 4 & 6,8 & 0,1 & 3 & 2 \\ \hline
256 & 4 & 7,2 & 0,1 & 2 & 2 \\ \hline
284 & 4 & 8,0 & 0,1 & 0 & 2 \\ \hline
\end{tabular}\\
\textbf{Таблица 5.} Зависимость $I_a(B)$ для $V_a = (70 \pm 1)$ В.
\end{center}

\begin{center}
\begin{tabular}{|c|c|c|c|c|c|}
\hline
$I_m$, мА & $\sigma_{I_m}$, мА & $B$, мТл & $\sigma_B$, мТл & $I_a$, мкА & $\sigma_{I_a}$, мкА \\ \hline
0 & 4 & 0,0 & 0,1 & 254 & 2 \\ \hline
8 & 4 & 0,2 & 0,1 & 258 & 2 \\ \hline
36 & 4 & 1,0 & 0,1 & 254 & 2 \\ \hline
48 & 4 & 1,3 & 0,1 & 254 & 2 \\ \hline
60 & 4 & 1,7 & 0,1 & 254 & 2 \\ \hline
76 & 4 & 2,1 & 0,1 & 254 & 2 \\ \hline
84 & 4 & 2,4 & 0,1 & 252 & 2 \\ \hline
92 & 4 & 2,6 & 0,1 & 250 & 2 \\ \hline
104 & 4 & 2,9 & 0,1 & 238 & 2 \\ \hline
108 & 4 & 3,0 & 0,1 & 226 & 2 \\ \hline
116 & 4 & 3,2 & 0,1 & 220 & 2 \\ \hline
120 & 4 & 3,4 & 0,1 & 216 & 2 \\ \hline
128 & 4 & 3,6 & 0,1 & 214 & 2 \\ \hline
134 & 4 & 3,8 & 0,1 & 220 & 2 \\ \hline
140 & 4 & 3,9 & 0,1 & 220 & 2 \\ \hline
152 & 4 & 4,3 & 0,1 & 202 & 2 \\ \hline
158 & 4 & 4,4 & 0,1 & 200 & 2 \\ \hline
162 & 4 & 4,5 & 0,1 & 192 & 2 \\ \hline
170 & 4 & 4,8 & 0,1 & 184 & 2 \\ \hline
180 & 4 & 5,0 & 0,1 & 168 & 2 \\ \hline
192 & 4 & 5,4 & 0,1 & 120 & 2 \\ \hline
196 & 4 & 5,5 & 0,1 & 90 & 2 \\ \hline
204 & 4 & 5,7 & 0,1 & 50 & 2 \\ \hline
208 & 4 & 5,8 & 0,1 & 32 & 2 \\ \hline
212 & 4 & 5,9 & 0,1 & 24 & 2 \\ \hline
216 & 4 & 6,0 & 0,1 & 20 & 2 \\ \hline
224 & 4 & 6,3 & 0,1 & 14 & 2 \\ \hline
232 & 4 & 6,5 & 0,1 & 10 & 2 \\ \hline
242 & 4 & 6,8 & 0,1 & 6 & 2 \\ \hline
256 & 4 & 7,2 & 0,1 & 4 & 2 \\ \hline
268 & 4 & 7,5 & 0,1 & 3 & 2 \\ \hline
284 & 4 & 8,0 & 0,1 & 2 & 2 \\ \hline
300 & 4 & 8,4 & 0,1 & 0 & 2 \\ \hline
\end{tabular}\\
\textbf{Таблица 6.} Зависимость $I_a(B)$ для $V_a = (80 \pm 1)$ В.
\end{center}

\begin{center}
\begin{tabular}{|c|c|c|c|c|c|}
\hline
$I_m$, мА & $\sigma_{I_m}$, мА & $B$, мТл & $\sigma_B$, мТл & $I_a$, мкА & $\sigma_{I_a}$, мкА \\ \hline
0 & 4 & 0,0 & 0,1 & 262 & 2 \\ \hline
16 & 4 & 0,4 & 0,1 & 260 & 2 \\ \hline
44 & 4 & 1,2 & 0,1 & 258 & 2 \\ \hline
52 & 4 & 1,5 & 0,1 & 260 & 2 \\ \hline
68 & 4 & 1,9 & 0,1 & 260 & 2 \\ \hline
84 & 4 & 2,4 & 0,1 & 260 & 2 \\ \hline
104 & 4 & 2,9 & 0,1 & 256 & 2 \\ \hline
112 & 4 & 3,1 & 0,1 & 240 & 2 \\ \hline
116 & 4 & 3,2 & 0,1 & 234 & 2 \\ \hline
120 & 4 & 3,4 & 0,1 & 226 & 2 \\ \hline
128 & 4 & 3,6 & 0,1 & 226 & 2 \\ \hline
140 & 4 & 3,9 & 0,1 & 218 & 2 \\ \hline
144 & 4 & 4,0 & 0,1 & 208 & 2 \\ \hline
152 & 4 & 4,3 & 0,1 & 230 & 2 \\ \hline
164 & 4 & 4,6 & 0,1 & 220 & 2 \\ \hline
174 & 4 & 4,9 & 0,1 & 198 & 2 \\ \hline
180 & 4 & 5,0 & 0,1 & 190 & 2 \\ \hline
186 & 4 & 5,2 & 0,1 & 184 & 2 \\ \hline
190 & 4 & 5,3 & 0,1 & 180 & 2 \\ \hline
196 & 4 & 5,5 & 0,1 & 166 & 2 \\ \hline
200 & 4 & 5,6 & 0,1 & 158 & 2 \\ \hline
204 & 4 & 5,7 & 0,1 & 142 & 2 \\ \hline
208 & 4 & 5,8 & 0,1 & 108 & 2 \\ \hline
212 & 4 & 5,9 & 0,1 & 84 & 2 \\ \hline
216 & 4 & 6,0 & 0,1 & 54 & 2 \\ \hline
222 & 4 & 6,2 & 0,1 & 34 & 2 \\ \hline
228 & 4 & 6,4 & 0,1 & 24 & 2 \\ \hline
234 & 4 & 6,6 & 0,1 & 18 & 2 \\ \hline
240 & 4 & 6,7 & 0,1 & 12 & 2 \\ \hline
248 & 4 & 6,9 & 0,1 & 10 & 2 \\ \hline
264 & 4 & 7,4 & 0,1 & 6 & 2 \\ \hline
278 & 4 & 7,8 & 0,1 & 4 & 2 \\ \hline
288 & 4 & 8,1 & 0,1 & 2 & 2 \\ \hline
300 & 4 & 8,4 & 0,1 & 1 & 2 \\ \hline
312 & 4 & 8,7 & 0,1 & 0 & 2 \\ \hline
\end{tabular}\\
\textbf{Таблица 7.} Зависимость $I_a(B)$ для $V_a = (90 \pm 1)$ В.
\end{center}

\begin{center}
\begin{tabular}{|c|c|c|c|c|c|}
\hline
$I_m$, мА & $\sigma_{I_m}$, мА & $B$, мТл & $\sigma_B$, мТл & $I_a$, мкА & $\sigma_{I_a}$, мкА \\ \hline
0 & 4 & 0,0 & 0,1 & 264 & 2 \\ \hline
14 & 4 & 0,4 & 0,1 & 266 & 2 \\ \hline
40 & 4 & 1,1 & 0,1 & 264 & 2 \\ \hline
64 & 4 & 1,8 & 0,1 & 266 & 2 \\ \hline
72 & 4 & 2,0 & 0,1 & 268 & 2 \\ \hline
80 & 4 & 2,2 & 0,1 & 268 & 2 \\ \hline
96 & 4 & 2,7 & 0,1 & 264 & 2\\ \hline
108 & 4 & 3,0 & 0,1 & 264 & 2 \\ \hline
120 & 4 & 3,4 & 0,1 & 250 & 2 \\ \hline
130 & 4 & 3,6 & 0,1 & 238 & 2 \\ \hline
144 & 4 & 4,0 & 0,1 & 232 & 2 \\ \hline
156 & 4 & 4,4 & 0,1 & 248 & 2 \\ \hline
164 & 4 & 4,6 & 0,1 & 242 & 2 \\ \hline
168 & 4 & 4,7 & 0,1 & 236 & 2 \\ \hline
180 & 4 & 5,0 & 0,1 & 214 & 2 \\ \hline
192 & 4 & 5,4 & 0,1 & 204 & 2 \\ \hline
200 & 4 & 5,6 & 0,1 & 198 & 2 \\ \hline
208 & 4 & 5,8 & 0,1 & 178 & 2 \\ \hline
212 & 4 & 5,9 & 0,1 & 158 & 2 \\ \hline
216 & 4 & 6,0 & 0,1 & 150 & 2 \\ \hline
224 & 4 & 6,3 & 0,1 & 98 & 2 \\ \hline
230 & 4 & 6,4 & 0,1 & 50 & 2 \\ \hline
238 & 4 & 6,7 & 0,1 & 28 & 2 \\ \hline
248 & 4 & 6,9 & 0,1 & 20 & 2 \\ \hline
260 & 4 & 7,3 & 0,1 & 11 & 2 \\ \hline
272 & 4 & 7,6 & 0,1 & 8 & 2 \\ \hline
280 & 4 & 7,8 & 0,1 & 6 & 2 \\ \hline
292 & 4 & 8,2 & 0,1 & 4 & 2 \\ \hline
300 & 4 & 8,4 & 0,1 & 3 & 2 \\ \hline
\end{tabular}\\
\textbf{Таблица 8.} Зависимость $I_a(B)$ для $V_a = (100 \pm 1)$ В.
\end{center}

\begin{center}
\begin{tabular}{|c|c|c|c|c|c|}
\hline
$I_m$, мА & $\sigma_{I_m}$, мА & $B$, мТл & $\sigma_B$, мТл & $I_a$, мкА & $\sigma_{I_a}$, мкА \\ \hline
0 & 4 & 0,0 & 0,1 & 268 & 2 \\ \hline
20 & 4 & 0,6 & 0,1 & 270 & 2 \\ \hline
40 & 4 & 1,1 & 0,1 & 264 & 2 \\ \hline
54 & 4 & 1,5 & 0,1 & 266 & 2 \\ \hline
68 & 4 & 1,9 & 0,1 & 270 & 2 \\ \hline
88 & 4 & 2,5 & 0,1 & 268 & 2 \\ \hline
114 & 4 & 3,2 & 0,1 & 266 & 2 \\ \hline
140 & 4 & 3,9 & 0,1 & 244 & 2 \\ \hline
156 & 4 & 4,4 & 0,1 & 244 & 2 \\ \hline
168 & 4 & 4,7 & 0,1 & 254 & 2 \\ \hline
172 & 4 & 4,8 & 0,1 & 250 & 2 \\ \hline
180 & 4 & 5,0 & 0,1 & 242 & 2 \\ \hline
188 & 4 & 5,3 & 0,1 & 224 & 2 \\ \hline
196 & 4 & 5,5 & 0,1 & 214 & 2 \\ \hline
208 & 4 & 5,8 & 0,1 & 200 & 2 \\ \hline
212 & 4 & 5,9 & 0,1 & 200 & 2 \\ \hline
220 & 4 & 6,2 & 0,1 & 186 & 2 \\ \hline
224 & 4 & 6,3 & 0,1 & 166 & 2 \\ \hline
230 & 4 & 6,4 & 0,1 & 128 & 2 \\ \hline
236 & 4 & 6,6 & 0,1 & 88 & 2 \\ \hline
240 & 4 & 6,7 & 0,1 & 58 & 2 \\ \hline
244 & 4 & 6,8 & 0,1 & 40 & 2 \\ \hline
248 & 4 & 6,9 & 0,1 & 32 & 2 \\ \hline
256 & 4 & 7,2 & 0,1 & 22 & 2 \\ \hline
262 & 4 & 7,3 & 0,1 & 20 & 2 \\ \hline
268 & 4 & 7,5 & 0,1 & 14 & 2 \\ \hline
278 & 4 & 7,8 & 0,1 & 10 & 2 \\ \hline
284 & 4 & 8,0 & 0,1 & 8 & 2 \\ \hline
292 & 4 & 8,2 & 0,1 & 8 & 2 \\ \hline
\end{tabular}\\
\textbf{Таблица 9.} Зависимость $I_a(B)$ для $V_a = (110 \pm 1)$ В.
\end{center}

\newpage
Запишем данные из вышеприведенных таблиц в один график
\begin{center}
\includegraphics[width = \textwidth]{images/image7.png}\\
\textbf{График 5. } График для определения $B_{\text{кр}}$ в зависимости от $V_a$.
\end{center}

По этому графику мы получаем зависимость $B_{\text{кр}}^2$ от $V_a$.
\begin{center}
\begin{tabular}{|c|c|}
\hline
$B_{\text{кр}}^2$, $\cdot 10^{-5}$ Тл$^2$ & $V_a$, В \\ \hline
2,3 & 70 \\ \hline
2,7 & 80 \\ \hline
3,25 & 90 \\ \hline
3,72 & 100 \\ \hline
4,1 & 110 \\ \hline
\end{tabular}\\
\textbf{Таблица 10.} $B_{\text{кр}}^2$ от $V_a$
\end{center}

По этим данным построим график.
\begin{center}
\includegraphics[width = \textwidth]{images/image8.png}\\
\textbf{График 6.} График зависимости $B_{\text{кр}}^2$ от $V_a$.
\end{center}
По этим данным мы получаем 
\[\dfrac{e}{m} = (1,3 \pm 0,5) \cdot 10^{11} \text{Кл}/\text{кг}\]

\begin{center}
    \raggedleft
        \underline{\underline{\LARGE {Вывод}}}
\end{center}

В ходе данной работы было измерено удельное значение заряда электрона двумя методами. Как показали расчеты, метод с магнитной фокусировкой даёт более точный результат. Это связано с более простым устройством поля в устоновке с электронным осциллографом, где его можно всюду считать постоянным, в случае же с двухэлектродной вакуумной лампой такого эффекта добиться нельзя ввиду зависимости напряженности от расстояния до оси анода. Также 2й метод основан на сглаживании измерений - кривой \(I_{A}(B)\), что в принципе ведет к большим погрешностям, чем в исселедовании с помощью линейной аппроксимации
\end{document}
