\documentclass{article}
\usepackage[utf8]{inputenc}
\usepackage[russian]{babel}
\usepackage{amsmath}
\usepackage[left=4cm, right=4cm, top=2cm]{geometry}
\usepackage{array}
\usepackage{amssymb}
\usepackage{graphicx}
\usepackage{ragged2e}
\usepackage{wrapfig}
\justifying


\title{
    \textbf{Лабораторная работа 2.2.1.}
}
\author{Герасименко Д.В.}
\date{1 курс ФРКТ, группа Б01-104}

\begin{document}

\maketitle

\begin{center}
    \raggedleft
    {
        \LARGE {Аннотация}
    }
    \hline
    \hline
\end{center}

\begin{center}
    \raggedright
    {
        \Large{\textbf{Тема:}}
        \\
    }
    \large {Исследование взаимной диффузии газов}
\end{center}

\begin{center}
    \raggedright
    {
        \large{\textbf{Цели работы:}}
        \\
    }
    \large {1) Определение коэффициента диффузии по результатам измерения}
    \\
    \large {2) Исследование явления взаимной диффузии газов}
    \\
    \large {3) Регистрация зависимости концентрации гелия в воздухе с помощью датчиков теплопроводности при разных начальных давлениях смеси газов}
    \\
\end{center}

\begin{center}
    \raggedright
    {
        \large{\textbf{Необходимое оборудование:}}
        \\
    }
    \large {Измерительная установка, форвакуумный насос, баллон с газом, манометр, источник питания, цифровой омметр, секундомер}
\end{center}

\begin{center}
    \raggedleft
    {
        \LARGE {Теория}
    }
    \hline
    \hline
\end{center}

\begin{center}
    \raggedright
    {
        \large{\underline{Основное уравнение диффузии}}
    }
\end{center}

Диффузия - самопроизвольное перемешивание молекул, происходящее вследствие их хаотичного теплового движения, если в системе есть молекулы разного сорта, то говорят о \textit{взаимной} диффузии. Для исследования диффузии необходимо строгое соблюдение ламинарности потоков, что достигается при равенстве давлений частей смеси.

В системе с диффузией выполняется закон Фика, согласно которому градиент концентрации вещества системы пропорциональен потоку вещества в данном объеме:

\begin{equation}
    J = - D \nabla N
\end{equation}

Где \(D\) - коэффициент диффузии, \(J\) - диффузионный потока вещества, \(N\) - концентрация её в данном объёме. В одномерно случае, если происходит взаимная диффузия двух компонент \(a\) и \(b\), то уравнение (1) для каждой из компонент преобразуется как:

\begin{equation}
    j_{a} = - D_{ab} \cdot \frac{\partial n_{a}}{\partial x}; \;
    j_{b} = - D_{ba} \cdot \frac{\partial n_{b}}{\partial x}
\end{equation}

В данной работе предполагается поддержание неизменными температуры и давления смеси, откуда из основного уравенения состояния идеального газа получаем: \(P = (n_{He} + n_{D})\cdot kT \Rightarrow \Delta n_{He} = - \Delta n_{D}\). Поэтому для описания диффузии будем рассматривать уравнение только для потока гелия. Также эксперимент будет проходить в условиях малой концентрации гелия по сравнению с концентрацией воздуха, а поэтому перемешивание газов можно рассматривать как диффузию легких частиц. Из параграфов 89-90 2го тома Сивухина находим, что вязкость потока жидкости подчиняется ньютоновскому закону вязкости:
\begin{equation}
    \eta = \frac{1}{3} nm\lambda\;\overline{v} = nm\; D
\end{equation}

Таким образом, коэффициент взаимной диффузии газов равен :

\begin{equation}
    D = \frac{1}{3} \lambda\;\overline{v}
\end{equation}

где \(\lambda = \frac{1}{\sigma n_{0}}\) - длина свободного пробега, а \(\overline{v} = \sqrt {\frac{8RT}{\pi \mu}}\) - средняя тепловая скорость частиц.

\begin{center}
    \raggedright
    {
        \large{\underline{Применение закона Фика в эксперименте}}
    }
\end{center}

В экспериментальной устновке есть два сосуда объёмами \(V_{1}\) и \(V_{2}\) соответственно. Они соединены трубкой малого объема по сравнению с объёмами сосудов, поэтому можно считать, что выравнивание концентрации происходит исключительно в трубке, а внутри сосудов устанавливается постоянной по объёму концентрация. Если бы \(n_{1}, n_{2} = const\), то \(J = const\) и следовательно:
\begin{equation}
    J = -DS\frac{n_{1} - n_{2}}{l}
\end{equation}
В предположении, что процесс выравнивания - медленный, можно считать, что ур-е (4) устанавливается в каждый момент времени. Допустим, за время \(\Delta t\) изменения концентрации в сосудах: \(\Delta n_{1}, \Delta n_{2}\). Тогда верно, что \(V_{1} \Delta n_{1} =- V_{2} \Delta n_{2} = J \Delta t\). Откуда получим:
\begin{equation}
    \frac{d(n_{1} - n_{2})}{n_{1} - n_{2}} = -\frac{D\;S}{l} \left(\frac{1}{V_{1}} + \frac{1}{V_{2}}\right) \cdot d\;t
\end{equation}
Откуда получим уравнение концентрации смеси от времени:
\begin{equation}
    \Delta n = \Delta n_{0} \cdot e^{- \frac{t}{\tau}}
\end{equation}
где \(\tau = \frac{V_{1} V_{2}}{V_{1} + V_{2}}\frac{l}{D\;S}\). Для проверки процесса на квазистационарность необходимо убедиться в том, что \(\tau\) много больше времени диффузии одной частицы вдоль трубки: \(\tau \gg \frac{l^{2}}{D}\)

\begin{center}
    \raggedleft
    {
        \LARGE {Метод и экспериментальная установка}
    }
    \hline
    \hline
\end{center}

Мы будем фиксировать изменения концентрации газов, следя за его теплопроводностью. Для этого понадобятся датчики, с помощью которых проследим зависимость сопротивления проволоки от теплопроводности газа. Схема датчиков изображена на рис.1.

\begin{wrapfigure}{l}{0.6\textwidth}
    \centering
    \includegraphics[width=0.25\textwidth]{image.tex/dozator.png}
    \includegraphics[width=0.25\textwidth]{image.tex/detector.png}

    \textit{рис.1.} Датчик теплопроводности и дозатор
\end{wrapfigure}


Для измерения сопротивлений используется мостовая схема, балансирующаяся при заполнении сосудов (и датчиков) одной и той же смесью. При перемешивании газов в сосудах возникает «разбаланс» моста. При незначительном различии в составах смесей показания вольтметра, подсоединённого к диагонали моста, будут пропорциональны разности концентраций примеси: \(U \sim \Delta C \sim \Delta (n_{1} - n_{2})\). В процессе диффузии разность концентраций убывает по закону (6), и значит по тому же закону изменяется напряжение:
\begin{equation}
    U = U_{0} \cdot e^{- \frac{t}{\tau}}
\end{equation}

Экспериментальная установка состоит из сосудов объёмами \(V_{1}, V_{2}\), соединительной трубкой с краном \(K_{3}\), в которой будет происходить диффузия, сообщающей трубкой с баллоном гелия, форвакуумного насоса и подачи атмосферы. Гелий будет запускаться в установку с помощью дозатора, устройство которого приведено на рис.1.

\begin{center}
    \includegraphics[width=0.6\textwidth]{image.tex/instalation.png}

    \textit{рис.2.} Схема установки
\end{center}

Датчики теплопроводности Д\(_{1}\) и Д\(_{2}\), расположенные в сосудах \(V_{1}\) и \(V_{2}\) соответственно, включены в мостовую электрическую схему согласно рис. 1. В одну из диагоналей моста включён высокочувствительный вольтметр (гальванометр) Г, к другой подключается источник небольшого постоянного напряжения. Сопротивления проволок датчиков составляют одно из плеч моста. Второе плечо составляют переменные сопротивления R1, R2 и R, служащие для установки показаний вольтметра Г на нуль (балансировка моста). Сопротивления R1 и R2 спарены (их подвижные контакты находятся на общей оси) и изменяются одновременно при повороте ручки грубой регулировки. Точная балансировка выполняется потенциометром R. Балансировку необходимо проводить перед каждым экспериментом заново: при этом установка заполняется чистым газом (воздухом без гелия) при давлении, близком «рабочему» (при котором затем будут проводится измерения)
\\
\begin{center}
    \raggedleft
    {
        \LARGE {Выполнение и обработка результатов}
    }
    \hline
    \hline
\end{center}

\begin{center}
    \raggedleft
    {
        \large{\underline{Экспериментальные погрешности и необходимые справочные данные}}
    }
\end{center}

Цена деления манометра: 199 делений соответствует атмосферному давлению в 750 мм рт.  ст. \(\Rightarrow\) цена деления: \(\sigma_{M} = 3,7\) торр \(\cdot\) дел \(^{-1}\).

Рабочие объемы сосудов: \(V_{1} = V_{2} = (252 \pm 1)\) см\(^{3}\); \(\varepsilon_{V} =  0,4\%\).

Отношение длины диффузорной трубки к площади её сечения:

\(\frac{L}{S} = (5,3 \pm 0,1)\) см \(^{-1}\); \(\varepsilon_{ls} = 1,8 \%\)

\begin{center}
    \raggedleft
    {
        \large{\underline{Выполнение}}
    }
\end{center}
Подготовим установку к работе:

1) Очистим установку от всех газов с помощью форвакуумного насоса до давления \(\sim 0,1 \) торр.

2) Напустим воздуха в установку из крана К\(_{5}\) до рабочего давления в \(P_{r} \sim 40\) торр. Сбалансируем мост датчиков.

3) Подадим гелий в установку, предварительно откачав воздух из патрубков с помощью форвакуумного насоса. Подачу осуществим порционно с помощью дозатора. По оканчании подачи плотно закроем кран К\(_{7}\).

4) Откачаем гелий из патрубков и подадим в сосуд с воздухом давление в \(1,5\cdot P_{r}\). Изолируем систему газов, перекрыв краны К\(_{1}\) и К\(_{2}\).

5) Откроем кран К\(_{3}\) и запустим программу на компьютере.

Проведем эксперимент при разных рабочих давлениях и результаты внесем в таблицы ниже. Построим графики зависимостей \(ln(\frac{U}{U_{0}})\) и по наклонну графиков определим значение коэффициента \(-\frac{1}{\tau}\)
\\
\\
\\
\\
\\
\\
\begin{center}
    \centering
    \includegraphics[width=1\textwidth]{image.tex/data.png}

    \textit{табл.} 1. Зависимость напряжения датчиков от времени
\end{center}

\begin{center}
    \centering
    \includegraphics[width=1\textwidth]{image.tex/u(t).png}
\end{center}

\begin{center}
    \centering
    \includegraphics[width=0.9\textwidth]{image.tex/ln(t).png}
\end{center}

Обработаем полученные результаты с помощью метода наименьших квадратов. Каждая из зависимостей - линейна, следовательно, выражается как: \(ln(\frac{U}{U_{0}}) = b \cdot t + a\). Запишем получившиеся коэффициенты для каждой из зависимостей.

\begin{center}
    \begin{tabular}{|m{7em}|m{5em}|m{5em}|m{3em}|m{3em}|}
        \hline
        Давление, торр & \(b, 10^{-3}\;c^{-1}\) & \(\sigma_{b}, 10^{-3}\;c^{-1}\) & \(a\) & \(\sigma_{a}\) \\
        \hline
        40  & -5,21 & 0,01 & 2,82 & 0,0006 \\
        \hline
        100 & -3,34 & 0,01 & 2,83 & 0,0008 \\
        \hline
        200 & -1,47 & 0,02 & 2,74 & 0,0012 \\
        \hline
        300 & -1,06 & 0,02 & 2,72 & 0,0012 \\
        \hline
    \end{tabular}

    \textit{табл. 2.} Для просчета \(- \frac{1}{\tau}\).
\end{center}

Теперь вычислим коэффициент диффузии гелия для каждого из давления. Оценим погрешность вычисляемой величины. Так как введенная \(\tau = \frac{V_{1} V_{2}}{V_{1} + V_{2}}\frac{l}{D\;S}\), то
\begin{equation}
    D = \frac{l}{S} \cdot \frac{V_{1}\; V_{2}}{V_{1} + V_{2}} \cdot \frac{1}{\tau}
\end{equation}
тогда относительная погрешность:
\begin{equation}
    \left(\frac{\sigma_{D}}{D}\right) = \sqrt {\varepsilon_{ls}^{2} + \varepsilon_{V}^{2} + \varepsilon_{b}^{2} }
\end{equation}

Данные внесем в таблицу:
\begin{center}
    \begin{tabular}{|m{7em}|m{6em}|m{6em}|}
        \hline
        Давление, торр & D, см\(^{2} \cdot\) c\(^{-1}\) & \(\sigma_{D}\), см\(^{2}\cdot\) c\(^{-1}\) \\
        \hline
        40  & 3,45 & 0,06 \\
        \hline
        100 & 1,63 & 0,03 \\
        \hline
        200 & 0,97 & 0,02 \\
        \hline
        300 & 0,71 & 0,01 \\
        \hline
    \end{tabular}
\end{center}

Теперь изобразим график зависимости \(D(\frac{P}{P_{0}})\), значение \(D_{0}\) при атмосферном давлении сравним с табличным. Из графика находим, что при \(\frac{1}{P} = \frac{1}{760} = 0,0013\): \(D_{0} = (0,151 \pm 0,005)\) см\(^{2} \cdot \) c\(^{-1}\), что отличается от табличного и не совпадает с ним в пределах посчитанной погрешности: \(D_{real} = 0,62\) см\(^{2} \cdot \) c\(^{-1}\).

\begin{center}
    \centering
    \includegraphics[width=0.9\textwidth]{image.tex/1dp.png}
\end{center}

Теперь оценим длину свободного пробега. По формуле (4) имеем:

\begin{equation}
    \lambda = 3D \cdot \sqrt{\frac{\pi\;\mu}{8\;R\;T}} \Rightarrow \varepsilon_{\lambda} = \sqrt{\varepsilon_{D}^{2} + \varepsilon_{k_{D\left(\frac{1}{P}\right)}}^{2}} = 8,4 \%
\end{equation}

\begin{center}
    \begin{tabular}{|m{8em}|m{2em}|m{2em}|m{2em}|m{2em}|m{2em}|}
        \hline
        Давление, торр  & 40  & 100 & 200 & 300 & 760\\
        \hline
        \(\lambda\), нм & 831 & 393 & 234 & 171 & 96 \\
        \hline
    \end{tabular}
\end{center}

Тогда длина свободного пробега гелия при атмосверном давлении:
\begin{center}
    \(\lambda_{0} = (96 \pm 8)\) нм
\end{center}

\begin{center}
    \raggedleft
    {
        \LARGE {Вывод и обсуждение результатов работы}
    }
    \hline
    \hline
\end{center}

1) Исследована взаимная диффузия на примере проникновения примеси легких частиц гелия.

2) Найдено значение коэффициента диффузии при экстраполяции к 760 торр, которое не совпадает с табличным в пределах погрешности.

3) Оценена длина свободного пробега гелия при различных давлениях. По порядку совпадает с эталлонным, но численно отличается в 2 раза

4) Проблемы в найденных результатах могут быть обусловлены неправильным трактованием объёма установки. Если бы он был в 2 раза больше, чем \(V_{1}\), то все найденные значения вошли бы в предел погрешности.

\end{document}
