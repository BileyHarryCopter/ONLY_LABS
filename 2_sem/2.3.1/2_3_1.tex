\documentclass{article}
\usepackage[utf8]{inputenc}
\usepackage[russian]{babel}
\usepackage{amsmath}
\usepackage[left=4cm, right=4cm, top=2cm]{geometry}
\usepackage{array}
\usepackage{amssymb}
\usepackage{graphicx}
\usepackage{ragged2e}
\usepackage{wrapfig}
\justifying


\title{
    \textbf{Лабораторная работа 2.3.1.}
}
\author{Герасименко Д.В.}
\date{1 курс ФРКТ, группа Б01-104}

\begin{document}

\maketitle

\begin{center}
    \raggedleft
    {
        \LARGE {Аннотация}
    }
    \hline
    \hline
\end{center}

\begin{center}
    \raggedright
    {
        \Large{\textbf{Тема:}}
        \\
    }
    \large {Получение и измерение вакуума}
\end{center}

\begin{center}
    \raggedright
    {
        \large{\textbf{Цели работы:}}
        \\
    }
    \large {1) Измерение объёмов форвакуумной и высоковакуумной частей установки}
    \\
    \large {2) Достижение высокого вакуума в \(\sim 10^{-5}\)}
    \\
    \large {3) Определение скорости откачки системы в стационарном режиме, а также по ухудшению и по улучшению вакуума}
\end{center}

\begin{center}
    \raggedright
    {
        \large{\textbf{Необходимое оборудование:}}
        \\
    }
    \large {Вакуумная установка с манометрами: масляными, термопарными и ионизационными}
\end{center}

\begin{center}
    \raggedleft
    {
        \LARGE {Теория}
    }
    \hline
    \hline
\end{center}

По степени разрежения вакуумные установки принято делить на три класса: низковакуумные — до \(10^{-2}\;–\;10^{−3}\) торр; высоковакуумные — \(10^{-4}\;-\;10^{-7}\) торр; установки сверхвысокого вакуума — \(10^{-8}\;-\;10^{-11}\) торр.

\begin{center}
    \raggedleft
    {
        \LARGE {Метод и экпериментальная установка}
    }
    \hline
    \hline
\end{center}

\begin{center}
    \raggedleft
    {
        \large{\underline{Метод}}
    }
\end{center}

Метод откачки состоит из двух фаз: фазы с форвакуумным насосом (откачивание до \(\sim 10^{-2}\) торр) и фазы совместного откачивания форвакуумного насоса и диффузионного (до \(\sim 10^{-5}\) торр).

\begin{center}
    \raggedleft
    {
        \large{\underline{Экспериментальная установка}}
    }
\end{center}

Плотности веществ: масло \(\rho_{m} = 940\) кг\(\cdot\) см \(^{-3}\); ртуть \(\rho_{hg} = 13550\) кг\(\cdot\) м \(^{-3}\)

Параметры трубки: \(d_{kap} = 0,8\) см; \(L = (10,8 \pm 0,1)\) см

Схема экспериментальной установки выглядит следующим образом:

\begin{center}
    \includegraphics[width=1.0\textwidth]{image.tex/sheme.png}

    \textit{Рис. 1.} Вакуумная установка.
\end{center}

\underline{Форвакуумный насос}

Действие насоса ясно из изображенных на рис. 2 последовательных положений пластин при вращении ротора по часовой стрелке. В положении «а» газ из откачиваемого объема поступает в пространство между пластиной «А» и линией соприкосновения корпуса и рото- ра. По мере вращения это пространство увеличивается (рис. 2 б), пока вход в него не перекроет другая пластина «Б» (рис. 2 в). После того как пластина «А» пройдет выходное отверстие и линию соприкосно- вения (рис. 2 г), лопасть «Б» будет сжимать следующую порцию газа и вытеснять его через клапан в атмосферу.

\begin{center}
    \includegraphics[width=1.0\textwidth]{image.tex/fvpump.png}

    \textit{Рис. 2.} Схема работы форвакуумного насоса.
\end{center}

\underline{Диффузионный насос}

Откачивающее действие диффузионного насоса состоит в следующем: попавшие в струю молекулы газа увлекаются ею и уже не возвращаются назад; на их месте образуется пустота, которая немедленно заполняется следующими порциями газа, увеличивая степень разрежения газа в окрестности струи.

Пары масла выходят за счет подогрева масла спиралью, по которой пущен переменный ток. Диффузионный насос, используемый в нашей установке (см. рис. 1), имеет две ступени и соответственно два сопла. Одно сопло вертикальное (первая ступень), второе сопло горизонтальное (вторая ступень). Легколетучие фракции масла, испаряясь, поступают в первую ступень, обогащая ее легколетучей фракцией масла. По этой причине плотность струи первой ступени выше и эта ступень начинает откачивать при более высоком давлении в форвакуумной части установки. Вторая ступень обогощается малолетучими фракциями. Плотность струи второй ступени меньше, но меньше и давление насыщенных паров масла в этой ступени. Соответственно в откачиваемый объем поступает меньше паров масла и его удается откачать до более высокого вакуума, чем если бы мы работали только с одной ступенью.

\underline {Термопарный манометр}

Чувствительным элементом манометра является платино-платинородиевая термопара, спаянная с никелевой нитью накала и заключенная в стеклянный баллон. По нити накала НН пропускается ток постоянной величины.

\begin{wrapfigure}{l}{0.4\textwidth}
    \centering
    \includegraphics[width=0.15\textwidth]{image.tex/termpar.png}

    \textit{рис.3.} Схема устройства термопарного манометра
\end{wrapfigure}

Для установки тока служит потенциометр R — «Рег. тока накала», расположенный па передней панели вакуумметра. Термопара ТТ присоединяется к милливольтметру1, показания которого определяются температурой нити накала и зависят от отдачи тепла в окружающее пространство.

Потери тепла определяются теплопроводностью нити и термопары, теплопроводностью газа, переносом тепла конвек- тивными потоками газа внутри лампы и теплоизлучением нити (инфракрасное тепловое излучение)

\underline {Ионизационный манометр}

Схема ионизационного манометра изображена на рисунке 4. Он представляет собой трехэлектродную лампу. Электроны испускаются накаленным катодом и увлекаются электрическим полем к аноду, имеющему вид редкой спирали. Проскакивая за ее витки, электроны замедляются полем коллектора и возвращаются к катоду, а от него вновь увлекаются к аноду. Прежде чем осесть на аноде, они успевают много раз пересечь пространство между катодом и кол- лектором. На своем пути электроны ионизуют молекулы газа. Ионы, образовавшиеся между анодом и коллектором, притягиваются полем коллектора и определяют его ток.

\begin{wrapfigure}{r}{0.5\textwidth}
    \centering
    \includegraphics[width=0.2\textwidth]{image.tex/ionmetr.png}

    \textit{рис.3.} Схема устройства термопарного манометра
\end{wrapfigure}


Ионный ток в цепи коллектора пропорционален плотности газа и поэтому может служить мерой давления.
\\

Накаленный катод ионизационного манометра перегорает, если давление в системе превышает 1 · 10−3 торр. Поэтому включать иони- зационный манометр можно, только убедившись по термопарному манометру, что давление в системе не превышает 10−3 торр

\begin{center}
    \raggedleft
    {
        \large{\underline{Процесс откачки}}
    }
\end{center}

Производиттельность насоса определяется скоростью откачки \(л / с\). В нашем случае основное уравнение процесса откачки принимает вид:

\begin{equation}
    -VdP = (PW - Q_{n} - Q_{d} - Q_{t})\;dt
\end{equation}

При достижении предельного вакуума реализуется случай:
\begin{equation}
    \frac{dP}{dt} = 0
\end{equation}

Из этого находим:
\begin{equation}
    W = \frac{\sum{Q_{i}}}{P}
\end{equation}

Проинтегрировав уравнение (1) получим экспоненциальную зависимость давления откачки от времени, с помощью которой найдем производтительеность насоса:

\begin{equation}
    P = P_{0} e^{-\frac{W}{V}t} + P_{pr}
\end{equation}

Закон сложения пропускных способностей аналогичен закону сложения проводимостей3. При последовательном соединении элементов:

\begin{equation}
    \frac{1}{W} = \frac{1}{W_{n}} + \sum \frac{1}{C_{i}}
\end{equation}
где \(C_{i}\) - пропускные способноси частей системы, а \(W_{n}\) - скорость собственной откачки насоса.

\begin{center}
    \raggedleft
    {
        \large{\underline{Течение через трубу}}
    }
\end{center}

Для количества газа, протекающего через трубу в условиях высокого вакуума или, как говорят, в кнудсеновском режиме, справедлива формула:

\begin{equation}
    \frac{d(PV)}{dt} = \frac{4}{3} r^{3} \sqrt{\frac{2 \pi R T }{\mu}} \cdot \frac{P_{2} - P_{1}}{l}
\end{equation}

Тогда пропускная способность трубы:
\begin{equation}
    C = \left(\fracc{dV}{dt}\right) = \frac{4 r^{3}}{3 L} \sqrt{\frac{2 \pi R T }{\mu}}
\end{equation}


\begin{center}
    \raggedleft
    {
        \LARGE {Выполнение и обработка результатов}
    }
    \hline
    \hline
\end{center}

\begin{center}
    \raggedleft
    {
        \large{\underline{Определение форвакуумной и высоковакуумной частей установки}}
    }
\end{center}

1) Откроем все краны установки и впустим атмосферный воздух через краны К\(_{1}\) и К\(_{2}\).

2) Закроем объём воздуха  между кранами К\(_{5}\) и К\(_{6}\). Включим форвакуумный насос и после установления давления \(\sim 10^{-2}\) торр отсоединим высоковакуумную часть от установки, перекрыв кран 3.

3) Закроем кран 4 и откроем кран 5. Измерим давление ранее запертого воздуха с помощью маслянистого манометра. Найдем объем форвакуумной части установки \(V_{fv}\).

4) Проведем те же манипуляции, не закрывая доступ воздуха к высоковакуной части установки. Найдя общий объёем установки и зная объём форвакуумной части найдем их разность и получим объём высоковакуумной части \(V_{vv}\).

Формула для вычисления объема частей установки:
\begin{equation}
    P_{0}\cdot V_{0} = P_{fv} \cdot V_{fv} ; \;
    P_{fv} = \frac{\rho_{oil}}{\rho_{hg}} \cdot \rho_{hg}\;g\;\Delta h_{fv}
\end{equation}
Аналогично для всего объема установки. Проведем измерения получим

Начальное давление: \(P_{0} = 756 торр\);

Начальный объём: \(V_{0} = (50,1 \pm 0,1)\) см\(^{3}\)

Достигнутое при откачивании давление: \(P_{vac} = 2\cdot10^{-2}\) мм рт ст;

Перепад уровней масла в форвакуумной части: \(\Delta h_{fv} = (268 \pm 1) \)мм

Установившееся в форвакуумной части давление: \(P_{fv} = (18,4 \pm 0,2)\) торр \(= (2432 \pm 5)\) Па

Объем форвакуумной части сосуда: \(V_{fv} = (2058 \pm 5)\) см \(^{3}\)

Перепад уровней масла сосуде: \(\Delta h_{fv} = (172 \pm 1) \)мм

Установившееся давление: \(P_{sum} = (11,9 \pm 0,2)\) торр \(= (1578 \pm 5)\) Па

Суммарный объём частей сосуда: \(V_{sum} = (3182 \pm 5)\) см \(^{3}\)

Объемы высоковакуумной части сосуда: \(V_{vv} = (1124 \pm 5)\) см \(^{3}\)

\begin{center}
    \raggedleft
    {
        \large{\underline{Получение высокого вакуума и измерение скорости откачки}}
    }
\end{center}

1) Откачаем установку форвакуумным насосом и включим термопары. Нагреем масло до кипения

2) Не выключая форвакуумного насоса, включим диффузионный насос при давлении \(\sim 10^{-2}\) торр.

3) Включим ионизационный манометр и запишем предельное достигнутое давление.

4) Зафиксируем зависимость давления от времени откачки и занесем данные в таблицы

\begin{center}
    \includegraphics[width=1\textwidth]{image.tex/vacuumUP.png}
\end{center}
\begin{center}
    \includegraphics[width=1\textwidth]{image.tex/vacuumDOWN.png}
\end{center}
\begin{center}
    \includegraphics[width=1\textwidth]{image.tex/grafUP.png}
\end{center}
\begin{center}
    \includegraphics[width=1\textwidth]{image.tex/grafDOWN.png}
\end{center}

Обработаем данные графиков по методу наименьших квадратов и найдем значение производительности системы \(W\). Результаты занесем в таблицу.

\begin{center}
    \begin{tabular}{|m{8em}|m{8em}|m{8em}|}
        \hline
        \(P_{pr}, 10^{-5}\) торр &  k_{down}, \(10^{-5}\)\(\cdot c ^{-1}\) & k_{up}, \(10^{-6}\)\(\cdot c ^{-1}\)\\
        \hline
        6,3 & \(0,12 \pm 0,01\) & \(-(0.15 \pm 0,01)\) \\
        \hline
        5,8 & \(0,17 \pm 0,01\) & \(-(0,17 \pm 0,01)\) \\
        \hline
        6,2 & \(0,16 \pm 0,01\) & \(-(0,15 \pm 0,01)\) \\
        \hline
    \end{tabular}

    \textit{Табл. 3.} Угловые коэффициенты прямых зависимостей
\end{center}

\newpage

\begin{center}
    \begin{tabular}{|m{10em}|m{5em}|m{5em}|m{5em}|}
    \hline
    Давление, \(10^{-5}\) торр & \(\frac{W}{V_{vv}}\), c\(^{-1}\) & W, л/с & \(\sigma_{W}\), л/с \\
    \hline
    6,3 & \(0,12 \pm 0,01\) & 0,135 & 0,011 \\
    \hline
    5,8 & \(0,17 \pm 0,01\) & 0,191 & 0,016 \\
    \hline
    6,2 & \(0,16 \pm 0,01\) & 0,179 & 0,015 \\
    \hline
    \end{tabular}
    \\
    \textit{Табл. 4.} Производительность откачки при ухудшении вакуума
\end{center}


\begin{center}
    \begin{tabular}{|m{10em}|m{7em}|m{5em}|m{5em}|}
    \hline
    Давление, \(10^{-5}\) торр & \(\frac{W}{V_{vv}}\), \(10^{-1}\) c\(^{-1}\) & W, л/с & \(\sigma_{W}\), л/с \\
    \hline
    6,3 & \(0,15 \pm 0,01\) & 0,017 & 0,001 \\
    \hline
    5,8 & \(0,17 \pm 0,01\) & 0,019 & 0,002 \\
    \hline
    6,2 & \(0,15 \pm 0,01\) & 0,017 & 0,001 \\
    \hline
    \end{tabular}

    \textit{Табл. 5.} Производительность откачки при улучшении вакуума
\end{center}

\begin{center}
    \raggedleft
    {
        \large{\underline{Оценка величины потока, поступающего из насоса назад в откачиваемую сиситему}}
    }
\end{center}

При закрытии крана 3 единственные потоки в высоковакуумной части - потоки десорбции с поверхности и через течи (\(Q_{d}\) и \(Q_{t}\) соответственно). Поэтому из основного уравнения, описывающего процесс откачки получим:
\begin{equation}
    V_{vv}\;dP = (Q_{d} + Q_{t})\;dt
\end{equation}

\begin{center}
    \raggedleft
    {
        \large{\underline{Оценка производительности насоса с помощью искусственной течи}}
    }
\end{center}

Зафиксируем предельное значение давления, после чего откроем кран 6, введя таким образом искусственную течь в систему. Измерим установившееся давление в форвакуумной части установки. Данные о капилярной трубке:

\begin{center}
    \(d_{kap} = 0,8\) см ; \; \(L = (10,8 \pm 0,1)\) см
\end{center}

Установившееся и предельное давления:

\begin{center}
    \(P_{pr} = (5,1 \pm 0,1) \cdot 10^{-5}\) торр ; \; \(P^{'} = (1,1 \pm 0,1) \cdot 10^{-4}\) торр
\end{center}

Тогда количество газа, протекающего через капилляр:

\begin{equation}
    \frac{d(PV)}{dt} = \frac{4}{3}\; r^{3} \sqrt{\frac{2\pi R\;T}{\mu}} \cdot \frac{P^{'}}{L}
\end{equation}

При закрытом и открытом кране 6 реализуются следующее пропускное распределение:

\begin{equation}
    P_{pr}\;W = Q_{1} ; \; P^{'}\;W = Q_{1} + \frac{d(PV)}{dt}
\end{equation}

Исключая натекание \(Q_{1}\), получим значение производительности:

\begin{equation}
    W = \frac{P^{'}}{P^{'} - P_{pr}} \cdot \frac{4 r^{3}}{3 L}\; \sqrt{\frac{2\pi R\;T}{\mu}} \approx \frac{4 r^{3}}{3 L}\; \sqrt{\frac{2\pi R\;T}{\mu}} = 0,018 \; l/c
\end{equation}
Соответственно погрешность которой можно оценить как: \(\varepsilon_{W} = \sqrt{\varepsilon_{L}^{2} + 2\cdot \varepsilon_{P}} = 14 \%\)

Откуда получаем значение производительности:
\begin{center}
    \(W = (0,018 \pm 0,003)\) л/с
\end{center}

Что входит в интервал значений производительности, полученных путем исследования улучшения вакуума.

\begin{center}
    \raggedleft
    {
        \LARGE {Вывод и обсуждение результатов работы}
    }
    \hline
    \hline
\end{center}

1) С достаточной точностью были определены объёмы всех частей вакуумной установки:

\begin{center}
    \(V_{fv} = (2058 \pm 5)\) см \(^{3}\) ;\; \(V_{vv} = (1124 \pm 5)\) см \(^{3}\)
\end{center}

2) Было найдено значение производительности насоса 2мя способами: с помощью установления искусственной течи и по изменению состояния вакуума. Значения, полученные обоими способам совпадают в пределах погрешности друг друга.

3) Подтвержден линейный характер зависимостей \(ln (P) (t)\).

4) Наибольшая достигнутая погрешность составляет \(14\%\), что для данной лабораторной работы можно считать успехом.


\end{document}
