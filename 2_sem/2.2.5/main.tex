\documentclass{article}
\usepackage[utf8]{inputenc}
\usepackage[russian]{babel}
\usepackage{amsmath}
\usepackage[left=4cm, right=4cm, top=2cm]{geometry}
\usepackage{array}
\usepackage{amssymb}
\usepackage{graphicx}
\usepackage{ragged2e}
\usepackage{wrapfig}
\justifying


\title{
    \textbf{Лабораторная работа 2.2.5.}
}
\author{Герасименко Д.В.}
\date{1 курс ФРКТ, группа Б01-104}



\begin{document}

\maketitle

\begin{center}
    \raggedleft
    {
        \LARGE {Аннотация}
    }
    \hline
    \hline
\end{center}

\begin{center}
    \raggedright
    {
        \Large{\textbf{Тема:}}
        \\
    }
    \large {Определение вязкости жидкости по скорости истечения через капилляр}
\end{center}

\begin{center}
    \raggedright
    {
        \large{\textbf{Цели работы:}}
        \\
    }
    \large {1) Исследование стационарного потока жидкости}
    \\
    \large {2) Определение вязкости воды с помощью измерения величины потока через капилляр. Применение формулы Пуазейля}
    \\
    \large {3) Определение вязкости других жидкостей по значению коэффициента вязкости воды}
\end{center}

\begin{center}
    \raggedright
    {
        \large{\textbf{Необходимое оборудование:}}
        \\
    }
    \large {Сосуд Мариотта, капиллярная трубка, мензурка, мерный стакан, секундомер, микроскоп на стойке}
\end{center}

\begin{center}
    \raggedleft
    {
        \LARGE {Теория}
    }
    \hline
    \hline
\end{center}

\begin{center}
    \raggedright
    {
        \large{\underline{Вывод формулы Пуазейля}}
    }
\end{center}

Рассмотрим трубку стационарного тока жидкости. Для нахождения распределения скоростей слоев жидкости в зависимости от расстояния до оси симметрии трубки предположим, что её радиус - \(R\). Мысленно выделим цилиндр жидкости радиусом \(r\), длиной \(l\), давление на концах которого - \(P_{1}\) и \(P_{2}\).

\begin{wrapfigure}{l}{0.25\textwidth}
    \centering
    \includegraphics[width=0.25\textwidth]{images.tex/puasel.png}

    рис.1. К выводу формулы Пуазейля
\end{wrapfigure}

Ввиду стационарности потока сила давления на торцы цилиндра должна компенсироваться силой трения на наружние слои цилиндра:
\begin{equation}
    \overrightarrow{F_{fric}} + \overrightarrow{F_{pres}} = \overrightarrow{0}
\end{equation}

где каждая из сил (трения и давления соответственно) с учетом направления вдоль оси тока:
\begin{center}
    \(F_{fric} = S_{||} \eta\frac{dv}{dr} = 2\pi r l \eta\frac{dv}{dr}\)
\end{center}
\begin{center}
    \(F_{pres} = S_{\perp}\Delta P = \pi r^{2} (P_{1} - P_{2})\)
\end{center}

Подставляя выражения для сил в уравнение (1) и интегрируя полученное равенство с учетом того, что у поверхности капилляра скорость жидкости равна нулю, получим искомую зависимость:

\begin{equation}
    v(r) = \frac{P_{1} - P_{2}}{4\eta l}(R^{2} - r^{2})
\end{equation}

Расход жидкости найдем как проходящий объем через сечение капилляра в единицу времени:
\begin{equation}
    Q = \frac{dV}{dt} = \int_{0}^{R} v 2 \pi rdr = \pi \frac{P_{1} - P_{2}}{8 \eta l} R^{4}
\end{equation}

\begin{center}
    \raggedright
    {
        \large{\underline{Число Рейнольдса}}
    }
\end{center}

Характер течения жидкости зависит от отношения кинетической энергии движения слоев к работе сил вязкости. При достаточно большом значении отношения характер течения переходит в турбулентный - силы вязкости не способны удерживать соседние слои от перемешивания.

Число Рейнольдса в гидродинамике определяется методом подобия. Кинетическая энергия некоторого объёма жидкости равна
\[K_{\delta V} = \rho L^{3} v^{2}\]
где \(L -\) характерный размер жидкости, который зависит от конкретной задачи. А работа сил трения, действующих на этот объём \[\delta A = -F_{fric} L = \eta \frac{v}{L} L^{2} L = \eta v L^{2}\]

А их отношение:
\begin{equation}
    R = \frac{\rho v L}{\eta}
\end{equation}

Для жидкости, двигающейся в трубке радиуса\(R\), характерный размер будет равен радиусу трубы. В гладких трубах круглого сечения переход от ламинарного течения к турбулентному происходит при: \(R \geqslant 1000.\)

\underline{Важность числа Рейнольдса} при данном эксперименте заключается в том, что с помощью него можно оценить расстояние, на котором от отверстия в сосуде до точки в капилляре устанавливается ламинарное течение:
\begin{equation}
    a \approx 0,2 R\cdot Re
\end{equation}

\begin{center}
    \raggedleft
    {
        \LARGE {Методы проведения эксперимента}
    }
    \hline
    \hline
\end{center}

\begin{center}
    \large{\underline{ЧАСТЬ А. Измерение вязкости воды}}
\end{center}

Для измерения вязкости воды воспользуемся сосудом Мариотта, схема конструкции котоорого представлена на рисунке ниже.
\begin{center}
    \includegraphics[width=0.4\textwidth]{images.tex/mariott.png}

    рис.2. Сосуд Мариотта
\end{center}

Сосуд Мариотта позволяет достигнуть постоянной разности давления в жидкости между точками B и A, чем будет обусловлен простой рассчет потока воды.

В установившемся режиме течения разность давлений между точками B и A не равна добавочному давлению столба воды высотой \(h\), так как необходимо сделать поправку на учет сил поверхностного натяжения. Перемещая калиброванную трубки вертикально, достигнем момента, когда пузырьки воздуха перестанут входить внутрь сосуда - соответствует случаю компенсации давления воздуха касательным напряжением поверхностного натяжения.

Внеся эту поправку в \(\Delta P = (h - \Delta h) \rho g\), получим линейную зависимость потока от высоты h. По данным из графика найдем коэффициент пропорциональности и вычислим значение коэффициента вязкости воды.

\begin{center}
    \large{\underline{ЧАСТЬ Б. Измерение вязкости водного раствора глицерина}} \large{\underline{вискозиметром Оствальда}}
\end{center}

С помощью вискозиметра Оствальда, имея значение коэффициента вязкости эталонной жидкости, можно найти коэффициент вязкости исследуемой жидкости.

\begin{wrapfigure}{l}{0.25\textwidth}
    \centering
    \includegraphics[width=0.25\textwidth]{images.tex/ostvald.png}

    рис.3. Вискозиметр Оствальда
\end{wrapfigure}
\\

Устройство вискозиметра заключается в следующем:

1) Вода, предварительно загнанная в узкую трубку с помощью резиновой груши, протекает через объём Ш\(_{1}\), за некоторое время \(t_{0}\)

2) Для расчета процесса течения жидкости в объёме Ш\(_{1}\) воспользуемся формулой Пуазейля в дифференциальной форме (рассмотрим значение потока в пределах небольших изменений параметров разности давления и объема воды)

3) Разность давления зависит от геометрии сосуда и однозначно определяется объёмом воды, находящейся внутри сосуда.

Тогда формула (2) принимает вид:

\begin{equation}
    -\frac{8l}{\pi R^{4} g} \cdot \frac{dV}{h(V)} = \frac{\rho}{\eta} \cdot dt
\end{equation}

Обратим внимание, что суммирование левой части равенства приведет к интегралу, имеющему постоянное значение для любой жидкости, так как определеятся исключительно геометрией сосуда. Тогда для иследуемых жидкостей должно быть верно:

\begin{equation}
    \frac{\rho}{\eta} t = const
\end{equation}

Проведя опыты сначала с водой, а потом с исследуемой жидкостью \(x\), получим:
\begin{equation}
    \eta_{x} = \eta_{0} \frac{\rho_{x}}{\rho_{0}} \cdot \frac{t_{x}}{t_{0}}
\end{equation}

\begin{center}
    \raggedleft
    {
        \LARGE {Выполнение и обработка результатов}
    }
    \hline
    \hline
\end{center}

Погрешности экспериментального оборудования :

1) \underline {Микроскоп 1:}  \indent  \indent \(\sigma_{mic1} = 0,005\) мм;

2) \underline {Микроском 2:}  \indent  \indent \(\sigma_{mic2} = 0,1\) мм;

3) \underline {Мерный стакан:} \indent  \(\sigma_{mc} = 1\) мм;

4) \underline {Секундомер:}   \indent  \indent \(\sigma_{c} = 0,1\) с

\newpage

\begin{center}
    \raggedleft
    {
        \large{\underline{Выполнение части А}}
    }
\end{center}

1) Измерим внутренний диаметр капилярной трубки сосуда Мариотта с помощью микроскопа. Данные занесем в таблицу 1, в качестве значения диаметра возьмем среднее.

\begin{center}
    \begin{tabular}{|m{10em}|m{2em}|m{2em}|m{2em}|m{2em}|m{2em}|}
        \hline
        Номер, № & 1 & 2& 3 & 4 & 5 \\
        \hline
        Диаметр,\(10^{-1}\) мм & 8,91 & 8,92 & 8,92 & 8,91 & 8,91 \\
        \hline
    \end{tabular}
\end{center}
\begin{center}
    \textbf {Таблица 1.} Измерение внутренного диаметра капиляра
\end{center} \\
Диаметр капилярной трубки: \(D = (8,91 \pm 0,05)\cdot 10^{-1}\) мм \\
Длина капилярной трубки: \(L = 137\) мм (указано на стойке сосуда) \\

2) После подготовления установки к эксперименту, измерим поправку \(\Delta h\) для правильного учета разности давления. Поднимем калиброванную вертикальную трубку до такого положения, при котором воздух не проникает внуть сосуда - не образуются пузыри. Измерим расстояние от оси капиляра трубки до нижнего торца трубки \textbf{B}: \(\Delta h = (20,9 \pm 0.1)\) мм

3) При разных значениях \(h\) измерим время вытекания объёма \(\Delta V = (20 \pm 0,5)\) мм\(^{3}\) из сосуда, также запишем поток. Данные занесем в таблицу 2:

\begin{center}
    \begin{tabular}{|m{7em}|m{3em}|m{3em}|m{3em}|m{3em}|}
        \hline
        Номер, №  & 1 & 2 & 3 & 4 \\
        \hline
        \(h\), мм & 59,8 & 70,1 & 80,2 & 90,1 \\
        \hline
        \(\Delta t\), c & 298,1 & 233,7 & 194,6 & 177,1 \\
        \hline
        \(Q\), \(10 ^{-2}\) мм\(^{3}\)/ с & 6,7 & 8,6 & 10,2 & 11,3 \\
        \hline
    \end{tabular}
\end{center}
\begin{center}
    \textbf{Таблица 2.} Измерение потока при разных значениях \(h\)
\end{center}

4) Убедимся в справедливости применения формулы (3) с помощью подсчета числа Рейнольдса. Для оценки примем значение вязкости воды равным \(\eta = 10^{-3}\) Па\(\cdot\)с. В качестве характерной скорости течения возьмем среднее по потоку: \(v = \frac{\Delta P}{8 \eta l} R^{2}\). Тогда число Рейнольдса:

\[
    Re = \frac{\rho v R}{\eta} = \frac{\Delta P \rho R^{3}}{8 \eta^{2} l} = \frac{\Delta P \rho D^{3}}{64 \eta^{2} l} \approx 71,2 \ll 1000
\]
Что соответствует ламинарному потоку, который устанавливается на расстонии от входа в капиляр:
\[
    a \approx 0,2\cdot Re \cdot R \approx 6,3 mm \ll L
\]

5) Изобразим данные таблицы 2 в зависимости потока от высоты - \(Q(h)\) на графике. По формуле (3) данная зависимость должна быть линейной, следовательно, по наклону графика найдем коэффициент пропорциональности, по значению которого вычислим коэффициент вязкости.
\\
\\
РАСЧЕТ ПОГРЕШНОСТИ ИЗМЕРЕНИЯ...

\begin{center}
    \includegraphics[width=1\textwidth]{images.tex/grafpQ.png}

    \textbf{График 1.} Зависимость потока от высоты нижнего торца трубы над осью капиляра
\end{center}

\begin{center}
    \raggedleft
    {
        \large{\underline{Выполнение части B}}
    }
\end{center}

1) Проведем опыт для дистилированной воды: нальём её в вискозиметр Оствальда и по достижении отметки "0" начнем отсчет времени. По достижении отметки "1" закончим времени. Проведем 5 измерений. Аналогично сделаем для растворов глицерина с концентрацией 10\%, 20\% и 30\%. Данные занесу в таблицу 3:

\begin{center}
    \begin{tabular}{|m{10em}|m{3em}|m{3em}|m{3em}|m{3em}|m{3em}|}
        \hline
        Время, с | Номер, №  & 1 & 2 & 3 & 4 & 5 \\
        \hline
        Дистил. вода & 9,5 & 9,8 & 9,4 & 9,5 & 9,6 \\
        \hline
        р-р 10\% & 12,3 & 12,2 & 11,9 & 12,1 & 12,0 \\
        \hline
        р-р 20\% & 18,4 & 18,4 & 18,4 & 18,3 & 18,5 \\
        \hline
        р-р 30\% & 28,3 & 28,3 & 28,4 & 28,1 & 28,2 \\
        \hline
    \end{tabular}
\end{center}

\begin{center}
    \textbf{Таблица 3.} Время истечения разных жидкостей через Вискозиметр Оствальда
\end{center}

2) Определим плотности исследуемых жидкостей из справки, прикрепленной к стойке с сосудом:

\begin{center}
    \begin{tabular}{|m{5em}|m{9em}|}
        \hline
        Жидкость & Плотность, г/см^3 \\
        \hline
        10 \% & 1,0192 \\
        \hline
        20 \% & 1,0415 \\
        \hline
        30 \% & 1,0646 \\
        \hline
    \end{tabular}
\end{center}
\begin{center}
    \textbf{Таблица 4.} Плотности растворов
\end{center}

3) Используя формулу (8), посчитаем коэффициенты вязкости каждого из растворов
\\
\\
РАСЧЕТ ПОГРЕШНОСТИ ИЗМЕРЕНИЯ...

\end{document}
